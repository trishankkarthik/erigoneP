\documentclass[11pt]{article}
\usepackage{mathpazo}
\usepackage{url}
\usepackage{verbatim}

\newcommand*{\prg}{\textsc{Erigone}}
\newcommand*{\trc}{\textsc{Trace}}
\newcommand*{\spn}{\textsc{Spin}}
\newcommand*{\prm}{\textsc{Promela}}
\newcommand*{\ada}{\textsc{Ada 2005}}
\newcommand*{\gnat}{\textsc{gnat}}
\newcommand*{\smc}{\textit{SMC}}
\newcommand*{\p}[1]{\texttt{#1}}

\textwidth=15cm
\textheight=22cm
\topmargin=0pt
\headheight=0pt
\oddsidemargin=1cm
\headsep=0pt
\renewcommand{\baselinestretch}{1.1}
\setlength{\parskip}{0.20\baselineskip plus 1pt minus 1pt}
\parindent=0pt

\title{The \prg{} Model Checker\\\mbox{}\\\large{Version 1.1.4}}
\author{Mordechai (Moti) Ben-Ari\\
Department of Science Teaching\\
Weizmann Institute of Science\\
Rehovot 76100 Israel\\
\textsf{http://stwww.weizmann.ac.il/g-cs/benari/}}
%\date{}
\begin{document}
\maketitle
\thispagestyle{empty}

\vfill

\begin{center}
Copyright \copyright{} 2007-9 by Mordechai (Moti) Ben-Ari.
\end{center}
This work is licensed under the Creative Commons Attribution-ShareAlike 3.0
License. To view a copy of this license, visit
\url{http://creativecommons.org/licenses/by-sa/3.0/}; or, (b) send a letter
to Creative Commons, 543 Howard Street, 5th Floor, San Francisco,
California, 94105, USA.


\newpage

\section{Introduction}

\prg{} is a partial reimplementation of the \spn{} Model Checker. The goal
is to facilitate learning concurrency and model checking.
\begin{itemize}
\item \prg{} is single, self-contained, executable file so that
installation and use are trivial.
\item \prg{} produces a detailed trace of the model checking
algorithms. The contents of the trace are customizable and a uniform
keyword-based format is used that can be directly read or used by other
tools.
\item Extensive modularization is used in the design of the \prg{}
program to facilitate understanding the source code. This will also
enable researchers to easily modify and extend the program.
\end{itemize}
\prg{} implements a subset of \prm{} that is sufficient for
demonstrating the basic concepts of model checking for the verification
of concurrent programs. No language constructs are added so that
programs for \prg{} can be used with \spn{} when more expressiveness and
better performance are desired.

\prg{} is written in \ada{} for reliability, maintainability and
portability. No non-standard constructs are used.

The \prg{} software is copyrighted under the \textsc{GNU} General Public
License. The copyright statement and the text of the license are
included in the distribution archive.

This document is divided into two parts: the user's guide and
the documentation of the software structure.

\subsection*{Acknowledgements}
I would like to thank Gerard J. Holzmann for his generous assistance
throughout the development of this project. The compiler was developed
by Trishank Karthik Kuppusamy under the supervision of Edmond Schonberg.

\section{References}
\begin{itemize}
\item Christel Baier and Joost-Pieter Katoen.
\textit{Principles of Model Checking}. MIT Press, 2008. 
\item M. Ben-Ari. \textit{Ada for Software Engineers
(Second Edition with Ada 2005)}. Springer, 2009.
\item M. Ben-Ari. \textit{Principles of the Spin Model Checker}. Springer, 2008.
\item M. Ben-Ari. \textit{Principles of Concurrent and Distributed Programming (Second
Edition)}. Addison-Wesley, 2006.
\item Gerard J. Holzmann. \textit{The Spin Model Checker: Primer
and Reference Manual}.\\Addison-Wesley, 2004. The abbreviation
\smc{} is used to refer to this book.
\item Fred Kr\"{o}ger and Stephan Merz.
\textit{Temporal Logic and State Systems}. Springer, 2008.
\end{itemize}

\begin{tabular}{l@{\hspace{3em}}l}
\hline
\prg{} & \url{http://stwww.weizmann.ac.il/g-cs/benari/erigone/}\\
& \url{http://code.google.com/p/erigone/}\\
\gnat{} & \url{https://libre.adacore.com/}\\
\spn{} & \url{http://spinroot.com/}\\
\hline
\end{tabular}

\section{Installation, building and execution}
\begin{itemize}
\item \textbf{Installation}\\Download the archive \p{erigone-N.zip}
from Google Code. Open the archive
to a clean directory. The source code is in the archive
\p{erigone-source-N.zip}. The executable files are \p{erigone.exe}
and \p{trace.exe}. The subdirectories are:
\begin{itemize}
\item \p{docs}: documentation;
\item \p{examples}: \prm{} source code of example programs;
\item \p{src}: the source code of \prg{};
\item \p{trace}: the source code of the \trc{} program  (see
Section ~\ref{s.trace}).
\end{itemize}
The file \p{readme.txt} in \p{examples} describes the expected output
when verifying the \prm{} programs in the directory.

\item \textbf{Execution}
\begin{verbatim}
erigone [arguments] filename
\end{verbatim}

\item \textbf{File names}\\
If an extension is not given in \p{filename}, the default \prm{} source
file extension is \p{pml}. Other file names are obtained by using the
file root and predefined extensions: The result of compiling the program
is a file with extension \p{aut} that describes the automata . The trail
file extension is \p{trl}. If the LTL correctness property \p{filename} is
not given, the default is the root with extension \p{prp}.

\item \textbf{Build}\\\prg{} was built with the \gnat{} \ada{} compiler.
The source code is divided into several subdirectories and the \gnat{}
Project Manager is used to build the program. The file \p{erigone.gpr}
describes the project configuration. To build, execute:
\begin{verbatim}
gnatmake -Perigone [-O1] [-gnatn]
\end{verbatim}
Optimization level \p{-O1} significantly improves the performance of the
software; \p{-gnatn} enables inlining of subprograms.

Object files are placed in a subdirectory of \p{src} called \p{obj}; it
must exist before executing the project manager.
\end{itemize}

\section{Command-line arguments}
\textbf{Execution mode}:
\begin{center}
\begin{tabular}{|l|p{.45\textwidth}|}
\hline
\p{-r} & Random simulation\\\hline
\p{-i} & Interactive simulation\\\hline
\p{-g} & Guided simulation\\\hline
\p{-gN} & Guided simulation using the N'th trail\\\hline\hline
\p{-s} & Verification of safety\\\hline
\p{-a} & Verification of acceptance\\\hline
\p{-f} & Verification with fairness\\\hline\hline
\p{-b} & LTL to BA translation only\\\hline
\p{-c} & Compile \prm{} only\\\hline
\end{tabular}
\end{center}
The default is random simulation.

\textbf{Execution limits:}
Limits are set using the \p{-l} argument, followed by a character
identifying which limit is being set and the numerical value of the
limit:
\begin{center}
\begin{tabular}{|l|p{.4\textwidth}|r|}
\hline
\p{-lhN} & [H]ash slots & 22\\\hline
\p{-llN} & [L]ocation stack & 3\\\hline
\p{-lpN} & [P]rogress steps & 1\\\hline
\p{-lsN} & [S]tate stack & 2\\\hline
\p{-ltN} & [T]otal steps & 10\\\hline
\end{tabular}
\end{center}

The parameter \p{N} is in thousands (except for hash slots) and the defaults are listed in the
third column. For the stacks, the value is the number of \textbf{stack
frames} and not the number of \emph{bytes}. \emph{Total steps} is the
number of steps of a simulation or verification before the execution is
terminated. For a verification, a progress report can be printed after
every \emph{progress steps} have been executed if \p{g} is included in
the display arguments. The hash table is allocated $2^{N}$ slots, where
$16 \leq N \leq 32$.

\textbf{Additional execution arguments:}
\begin{center}
\begin{tabular}{|l|p{.45\textwidth}|}
\hline
\p{-h} & Display [h]elp screen\\\hline
\p{-mN} & Stop after the [m]'th error\\\hline
\p{-nN} & Seed of ra[n]dom simulation\\\hline
\p{-o} & Compiler l[o]gs\\\hline
\p{-t[f]} & Read L[T]L formula\\\hline
\end{tabular}
\end{center}
If \p{-m0} is used, \emph{all} errors are reported
and numbered trail files are written for each one.
The default seed is obtained from the clock. The compiler logs are
written to separate files, not to standard output. If a file name is not
given with the \p{-t} argument, the file name is constructed as
described above. There is no space between \p{-t} and the file name.

Accept labels have not yet been implemented, so \p{-t} should be set
with execution mode \p{-a} or \p{-f}; a warning message will be given if
this is not done.

\textbf{Display arguments:}\\
The argument \p{-d} displays all the data.
Selective display is possible with the argument
\p{-dX}, where \p{X} is a string of one or more of:

Compile and runtime messages:
\begin{center}
\begin{tabular}{|l|p{.45\textwidth}|}
\hline
\p{g} & Pro[g]ress messages\\\hline
\p{p} & [P]rogram transitions and symbols\\\hline
\p{r} & [R]un-time statistics\\\hline
\p{v} & [V]ersion and copyright notice\\\hline
\end{tabular}
\end{center}

Transitions:
\begin{center}
\begin{tabular}{|l|p{.45\textwidth}|}
\hline
\p{a} & [A]ll transitions from a state\\\hline
\p{c} & [C]hosen transition (simulation)\\\hline
\p{e} & [E]xecutable transitions from a state\\\hline
\p{l} & [L]ocation stack\\\hline
\p{t} & [T]rail\\\hline
\p{y} & B[y]te code\\\hline
\end{tabular}
\end{center}

States:
\begin{center}
\begin{tabular}{|l|p{.45\textwidth}|}
\hline
\p{h} & State [h]ash table\\\hline
\p{m} & States in a si[m]ulation\\\hline
\p{s} & [S]tate stack\\\hline
\p{u} & B[u]ffered channel contents\\\hline
\end{tabular}
\end{center}

LTL translation:
\begin{center}
\begin{tabular}{|l|p{.45\textwidth}|}
\hline
\p{b} & [B]\"{u}chi automaton\\\hline
\p{n} & [N]odes in the LT tableau\\\hline
\end{tabular}
\end{center}

\newpage

\section{The subset of \spn{} that is implemented}
\subsection*{\prm{} subset}
The subset of \prm{} currently implemented includes:
\begin{itemize}
\item Data types \p{byte}, \p{bit}, \p{bool}, \p{mtype}. Variables. Arrays.
\item \p{active proctype}, \p{active [N] proctype}
\item Expressions and assignment statements. \p{assert}.
\item \p{atomic}.
\item Guarded commands, labels, \p{break}, \p{skip}, \p{goto}.
\item \p{printf} with format specifiers \p{\%d}, \p{\%c}, \p{\%e} and an
optional final \verb+\n+.
\item Rendezvous and buffered channels. Send, receive and poll
statements.\\\p{len}, \p{empty}, \p{nempty} \p{full}, \p{nfull}.
\end{itemize}

The special meaning of labels beginning with \p{end}, \p{accepted}, and
\p{progress} is not supported; therefore, LTL formulas must be used for
liveness correctness claims and the valid end-states are only those
where all processes have terminated.

\subsection*{LTL formulas}
The syntax for LTL formulas is not precisely the same as in \spn{}.
Atomic propositions must start with a lower-case character and must not
contain the upper-case characters \p{'U'} and \p{'V'}. Defined atomic
propositions such as \verb+#define mutex (critical<=1)+ cannot be used
because there is no preprocessor. As a workaround, expressions can be
enclosed within delimiters: \verb+![]#(critical<=1)#+.

\subsection*{Size limits}
Limits on the size of the \prm{} models are compiled into \prg{} in
three package specifications: \p{Global} for basic declarations,
\p{Config} for table sizes and \p{Config\_State} for the size of the
compressed state vectors in the stack and hash table:
%(Table~\ref{tab.size}).
%\begin{table}[htbp]
\begin{center}
\begin{tabular}{|l|l|l|}\hline
\multicolumn{3}{|c|}{Declarations in package \p{Global}}\\\hline
Identifier & Type & Meaning\\\hline\hline
\p{Byte} & \p{mod 256} & Values and indices\\\hline
\p{Name} & \p{String(1..64)} & Names and source statements\\\hline
\multicolumn{3}{c}{}\\\hline
\multicolumn{3}{|c|}{Declarations in package \p{Config}}\\\hline
Identifier & Type & Meaning\\\hline\hline
\p{Process\_Index} & \p{Byte range 0..7} & Processes\\\hline
\p{Symbol\_Index} & \p{Byte range 0..31} & Symbols\\\hline
\p{Message\_Index} & \p{Byte range 0..3} & Elements in a channel message\\\hline
\p{Channel\_Index} & \p{Byte range 1..8} & Channels\\\hline
\p{Data\_Index} & \p{Byte range 0..31} & Bytes of data in a state\\\hline
\p{Transition\_Index} & \p{Byte range 0..127} & Transitions in a process\\\hline
\p{Location\_Index} & \p{Byte range 0..7} & Transitions from any state\\\hline
\p{Byte\_Code\_Index} & \p{Byte range 0..63} & Byte codes per statement\\\hline
\p{Interpret\_Index} & \p{Byte range 0..63} & Interpretation stack\\\hline
\p{Max\_Futures} & \p{constant := 4} & Number of future formulas\\\hline
\multicolumn{3}{c}{}\\\hline
\multicolumn{3}{|c|}{Declarations in package \p{Config\_State}}\\\hline
Identifier & Type & Meaning\\\hline\hline
\p{Process\_Size\_Index} & \p{Byte range 0..3} & Size of processes in a state vector\\\hline
\p{Variable\_Size\_Index} & \p{Byte range 0..15} & Size of variables in a state vector\\\hline
\end{tabular}
%\caption{Size limits}\label{tab.size}
\end{center}
%\end{table}
These limits are sufficient to verify the program in the examples
directory. The declaration of the size of the compressed state vector is
in a separate package \p{Config\_State} to enable a future
implementation of per-model verifiers with minimal recompilation.

\section{Simulation}
When a \prm{} model is simulated, there will usually be more than one
\emph{executable} transition in each state. The choice of which
transition to take can be \emph{random}, \emph{interactive}, or
\emph{guided} by the \emph{trail file} that is written during a
verification.

\section{Verification}
There are three modes of verification in \prg{}:
\begin{description}
\item[Safety] Checks safety properties;
\item[Acceptance] Checks for acceptance cycles;
\item[Fairness] Checks for acceptance cycles under weak fairness.
\end{description}
A correctness property can be specified by a formula in \emph{linear
temporal logic (LTL)}.

\section{Display format}

Data is displayed in a uniform format: a single line of named
associations, each of which is terminated with a comma. This format is
verbose, but easy to read and easy to parse. In the following
description, the display option needed to obtain each item is given in
parentheses and extra lines have been used to fit the page.

\subsection{Data structure display}

The display begins with a title line (\p{v}) and a line
with the parameters of the execution (\p{r}):

\begin{footnotesize}
\begin{verbatim}
Erigone v1.0.0, Copyright 2008-9 by Moti Ben-Ari, GNU GPL.
execution mode=simulation,simulation mode=random,seed=-1,total steps=10,
\end{verbatim}
\end{footnotesize}

The symbol table (\p{p}) includes the offset into the state vector of
each variable and the byte code (\p{y}) for initialization:

\begin{footnotesize}
\begin{verbatim}
symbol table start=,
variables=6,
type=byte_type,name=n,offset=0,scope=global_scope,
  byte code={iconst 0 0,byte_store 0 0,},
type=byte_type,name=finished,offset=1,scope=global_scope,
  byte code={iconst 0 0,byte_store 1 0,},
type=byte_type,name=P.i,offset=2,scope=local_scope,
  byte code={iconst 1 0,byte_store 2 0,},
type=byte_type,name=P.temp,offset=3,scope=local_scope,
  byte code={iconst 0 0,byte_store 3 0,},
type=byte_type,name=Q.i,offset=4,scope=local_scope,
  byte code={iconst 1 0,byte_store 4 0,},
type=byte_type,name=Q.temp,offset=5,scope=local_scope,
  byte code={iconst 0 0,byte_store 5 0,},
symbol table end=,
\end{verbatim}
\end{footnotesize}

Tables of numeric and string constants (if any) are displayed (\p{p});
the table of numberic constants is used to insert immediate operands into load
constant instructions \p{iconst}.

For each transition, the following data is displayed (\p{p}): the source
and target state, various flags, the source code and the byte code
(\p{y}):

\begin{footnotesize}
\begin{verbatim}
transitions start=,
processes=3,
process=P,initial=1,transitions=7,
number=0,source=1,target=3,atomic=0,end=0,accept=0,line=7,
  statement={(i> 10)},byte code={byte_load 2 0,iconst 10 0,icmpgt 0 0,},
number=1,source=1,target=4,atomic=0,end=0,accept=0,line=8,
  statement={else},byte code={logic_else 0 0,},
number=2,source=3,target=9,atomic=0,end=0,accept=0,line=13,
  statement={finished++},byte code={iinc 1 1,},
number=3,source=4,target=5,atomic=0,end=0,accept=0,line=9,
  statement={temp=n},byte code={byte_load 0 0,byte_store 3 0,},
number=4,source=5,target=6,atomic=0,end=0,accept=0,line=10,
  statement={n=temp+ 1},byte code={byte_load 3 0,iconst 1 0,iadd 0 0,byte_store 0 0,},
number=5,source=6,target=1,atomic=0,end=0,accept=0,line=11,
  statement={i++},byte code={iinc 2 1,},
number=6,source=9,target=0,atomic=0,end=1,accept=0,line=0,
  statement={},byte code={halt 0 0,},
process=Q,initial=1,transitions=7,
  ...
process=Finish,initial=1,transitions=3,
  ...
transitions end=,
\end{verbatim}
\end{footnotesize}

\subsection{LTL translation display}
The display of the translation of the LTL formula to a B\"{u}chi
automaton (BA) is done before that of the transitions of the processes,
because the transitions of the BA are added to the transition table as a
never claim.The display starts with the LTL formula together with the
formula obtained by pushing negation inward (\p{b}). This is followed by
the nodes of the tableau (\p{n}):

\enlargethispage{\baselineskip}

\begin{footnotesize}
\begin{verbatim}
ltl formula=![]<>csp,
push negation=<>[]!csp,
nodes start=,
expanding=,node=1,incoming={0,},new={<>[]!csp,},old=,next=,with set=,
expanding=,node=2,incoming={0,},new={[]!csp,},old={<>[]!csp,},next=,with set=,
expanding=,node=4,incoming={0,},new={!csp,},old={[]!csp,<>[]!csp,},next={[]!csp,},with set=,
expanding=,node=4,incoming={0,},new=,old={!csp,[]!csp,<>[]!csp,},next={[]!csp,},with set=,
expanding=,node=5,incoming={4,},new={[]!csp,},old=,next=,with set={4,},
expanding=,node=6,incoming={4,},new={!csp,},old={[]!csp,},next={[]!csp,},with set={4,},
expanding=,node=6,incoming={4,},new=,old={!csp,[]!csp,},next={[]!csp,},with set={4,},
  ...
exists=,node=6,new incoming={4,6,},
  ...
nodes end=,
\end{verbatim}
\end{footnotesize}

A BA is extracted from the tableau (\p{b}) and optimized:

\begin{footnotesize}
\begin{verbatim}
optimized buchi automaton start=,
source=0,target=0,atomic=0,end=0,accept=0,line=0,statement={1},
  byte code={iconst 1 0,},
source=0,target=4,atomic=0,end=0,accept=0,line=0,statement={!csp},
  byte code={bit_load 3 0,logic_not 0 0,},
source=4,target=4,atomic=0,end=0,accept=1,line=0,statement={!csp},
  byte code={bit_load 3 0,logic_not 0 0,},
optimized buchi automaton end=,
\end{verbatim}
\end{footnotesize}

The BA is used to create a never claim process in the transition table
(\p{p}). In addition, the set of accept states is identified to make
them easier to find.

\subsection{Display of a simulation}
A simulation begins from an initial state (\p{m}):

\begin{footnotesize}
\begin{verbatim}
initial state=,P=1,Q=1,R=1,n=0,finish=0,P.temp=0,Q.temp=0,
initial state=,P=1,Q=1,Finish=1,n=0,finished=0,P.i=1,P.temp=0,Q.i=1,Q.temp=0,
\end{verbatim}
\end{footnotesize}

For each step of the simulation, the display can contain the set of all
transitions from the state (\p{a}), the set of executable transitions
(\p{e}), the chosen transition (\p{c}) and the next state in the
simulation (\p{m}):

\begin{footnotesize}
\begin{verbatim}
all transitions=5,
process=P,source=1,target=3,atomic=0,end=0,accept=0,line=7,
  statement={(i> 10)},byte code={byte_load 2 0,iconst 10 0,icmpgt 0 0,},
process=P,source=1,target=4,atomic=0,end=0,accept=0,line=8,
  statement={else}, byte code={logic_else 0 0,},
process=Q,source=1,target=3,atomic=0,end=0,accept=0,line=19,
  statement={(i> 10)},byte code={byte_load 4 0,iconst 10 0,icmpgt 0 0,},
process=Q,source=1,target=4,atomic=0,end=0,accept=0,line=20,
  statement={else},byte code={logic_else 0 0,},
process=Finish,source=1,target=2,atomic=0,end=0,accept=0,line=29,
  statement={finished== 2},byte code={byte_load 1 0,iconst 2 0,icmpeq 0 0,},

executable transitions=2,
process=P,source=1,target=4,atomic=0,end=0,accept=0,line=8,
  statement={else},byte code={logic_else 0 0,},
process=Q,source=1,target=4,atomic=0,end=0,accept=0,line=20,
  statement={else},byte code={logic_else 0 0,},

chosen transition=,
process=Q,source=1,target=4,atomic=0,end=0,accept=0,line=20,
  statement={else},byte code={logic_else 0 0,},

next state=,P=1,Q=4,Finish=1,n=0,finished=0,P.i=1,P.temp=0,Q.i=1,Q.temp=0,
\end{verbatim}
\end{footnotesize}

To display the data in buffered channels use \p{-u}.
The first field is the current number of messages in the channel;
this followed by the messages themselves in brackets:

\begin{footnotesize}
\begin{verbatim}
next state=,Pinit=3,ch1=1,a=0,b=0,c=0,channel1={2,[1,2,3,],[4,5,6,],},
\end{verbatim}
\end{footnotesize}

The simulation can terminated by an assertion violation, or a valid or
invalid end state, or by exceeding the total steps allowed:

\begin{footnotesize}
\begin{verbatim}
simulation terminated=valid end state,
\end{verbatim}
\end{footnotesize}

The runtime statistics are then displayed (\p{r}):

\begin{footnotesize}
\begin{verbatim}
steps=87,
times=,compilation=0.33,simulation=0.18,
\end{verbatim}
\end{footnotesize}

\subsection{Display of a verification}
During a verification, the set of all transitions (\p{a}) and executable
transitions (\p{e}) is displayed as for a simulation. Operations on the
stacks can be displayed, both for the state stack (\p{s}):

\begin{footnotesize}
\begin{verbatim}
push state=10,p=6,q=6,wantp=1,wantq=1,critical=0,
  ...
top state=10,p=6,q=6,wantp=1,wantq=1,critical=0,
  ...
pop state=10,reason=no_more_transitions,
\end{verbatim}
\end{footnotesize}

and for the location stack (\p{l}):

\begin{footnotesize}
\begin{verbatim}
push transition=16,process=0,transition=5,never=0,visited=false,last=false,
  ...
top transition=16,process=0,transition=5,never=0,visited=true,last=false,
  ...
pop transition=16,
\end{verbatim}
\end{footnotesize}

An attempt is made to insert each new state into the hash table (\p{h});
it may succeed or fail:

\begin{footnotesize}
\begin{verbatim}
inserted=true,p=5,q=3,wantp=1,wantq=1,critical=1,
  ...
inserted=false,p=6,q=3,wantp=1,wantq=1,critical=0,
\end{verbatim}
\end{footnotesize}

For verification with acceptance, the inner flag will be displayed
indicating if the state is part of the outer or inner search; if the
verification includes fairness, the fairness counter is included,
and the ``seed'' state is also displayed:

\begin{footnotesize}
\begin{verbatim}
seed=,p=4,q=13,:never:=4,wantp=1,wantq=1,turn=2,csp=0,inner=1,fair=3,
  ...
inserted=true,p=14,q=11,:never:=4,wantp=1,wantq=1,turn=2,csp=1,inner=1,fair=1,
\end{verbatim}
\end{footnotesize}

The outcome and the runtime statistics (\p{r}) are
displayed at the end of a verification:

\begin{footnotesize}
\begin{verbatim}
verification terminated=never claim terminated,
steps=47,
state stack elements=9,element size=22,memory=198,
transition stack elements=14,element size=5,memory=70,
states stored=19,matched=9,total=28,element size=22,memory=418,
times=,compilation=0.03,verification=0.09,
\end{verbatim}
\end{footnotesize}

Progress messages can be displayed during a long verification (\p{g}).

\section{The \trc{} program}\label{s.trace}

The \trc{} program performs string processing on the output data written
by \prg{}. It prints the output of a simulation in tabular form. First
execute:
\begin{verbatim}
erigone -dcmp filename > filename.trc
\end{verbatim}
to write a file with the symbol table, states and transitions
taken. Then execute:
\begin{verbatim}
trace [arguments] filename
\end{verbatim}
The arguments are:
\begin{center}
\begin{tabular}{|l|l|}
\hline
\p{-tn} & Number of lines between column [t]itles\\\hline\hline
\p{-ln} & Column width for [l]ine numbers\\\hline
\p{-pn} & Column width for [p]rocesses\\\hline
\p{-sn} & Column width for [s]tatements\\\hline
\p{-vn} & Column width for [v]ariables\\\hline\hline
\p{-xs} & E[x]clude variables with \p{s}\\\hline
\p{-ms} & Exclude state[m]ents with \p{s}\\\hline
\end{tabular}
\end{center}
Here is an example of the output of this program:
\begin{footnotesize}
\begin{verbatim}
Trace v1.00, Copyright 2008-9 by Moti Ben-Ari, GNU GPL.
Proc Line Statement        wantp  wantq  critic 
                           0      0      0      
p    7    {!wantq}         0      0      0      
q    19   {!wantp}         0      0      0      
q    20   {wantq=true}     0      1      0      
p    8    {wantp=true}     1      1      0      
q    21   {critical++}     1      1      1      
p    9    {critical++}     1      1      2      
p    10   {assert(critical 1      1      2      
simulation terminated=assert statement is false,
\end{verbatim}
\end{footnotesize}

The \p{-x} and \p{-m} arguments are used for filtering the output.
The strings \p{s} are lists of strings terminated by \p{\#}:
\begin{verbatim}
-xwant# -massert#!#
\end{verbatim}

The meaning is that a column for a \emph{variable} (\p{-x}) or a row for
a \emph{statement} (\p{-m}) will not be printed if one of those strings
appears within the variable or statement, respectively. With the above
arguments, the columns for \p{wantp} and \p{wantq} will not be printed,
nor will the rows with the \p{assert} statements or the negation
operator.

To build \trc{} with \gnat{}, run:
\begin{verbatim}
gnatmake -gnat05 trace.adb
\end{verbatim}

\newpage

\begin{center}
\textbf{\Large Software structure}
\end{center}

\setcounter{section}{0}

\section{Directories}
\prg{} is a single program written in \ada{}. The main procedure is
\p{Erigone} in the source file \p{erigone.adb} contained directly in the
directory \p{src}. This directory also contains the package \p{Version} with
the copyright notice and the version number. The other source files are
organized in subdirectories:
\begin{itemize}
\item \p{compile}: The \prm{} compiler;
\item \p{data}: Data structures, including states, symbols, transitions, stacks, hash
table;
\item \p{execute}: Execution of the algorithms for simulation and
verification;
\item \p{general}: General code for IO, command-line arguments,
and configuration;
\item \p{ltl}: Translation of LTL formulas to B\"{u}chi automata.
\end{itemize}

Library-level variables are not initialized when they are declared, but
only in explicit initialization subprograms. This enables the program
to be run repeatedly---perhaps from a development environment---without
reloading. (This has not yet been implemented for the compiler.)

\section{Global and configuration data}

Global declarations are in package \p{Global}. All values and indices
are of type \p{Byte} (an 8-bit byte) so this places a natural limitation
on the number transitions, etc. The constant \p{None} is declared as the
last value of \p{Byte} and should not be used otherwise. Each identifier
is of the string subtype \p{Name} and is padded with blanks. The type
\p{Byte\_Array\_Base} is an array of \p{Byte}, indexed by \p{Byte}, that
is used for storing most data. The array is unconstrained to allow
components that are smaller arrays; the subtype \p{Byte\_Array} is
constrained to the full range of the index \p{Byte}.

The configuration data are described in the user's guide. Package
\p{Config} defines subtypes of \p{Byte} that are used as indices of
arrays storing data and can be changed to compile and execute larger
programs. Package \p{Config\_State} defines subtypes of \p{Byte} that
are used to constrain the size of a compressed state vector on the stack
and in the hash table (see \p{Compress\_Vectors}). They are declared in
a package separate from \p{Config} so that changing them will have a
minimal impact on the necessary recompilation. These constants are
compiled into \prg{}, though a future modication could compute them from
a \prm{} program as in \spn{}.

\section{Packages in the general subdirectory}
\begin{itemize}
\item \p{Byte\_IO} is an instantiation of
\p{Ada.Text\_IO.Modular\_IO} for bytes.
\item File extensions and a pointer to the file name are declared in \p{Files}.
\item \p{Random\_Numbers} constains an instantiation of
\p{Ada.Numerics.Float\_Random} and a function that converts these values to values of
type \p{Byte}.
\item Package \p{Options} contains the declarations of run-time options
such as the modes for \prg{}, the stack sizes and the flags controlling
what data is displayed.
\item The values of the options are set in \p{Arguments} which parses
the command line.
\item Package \p{Times} stores the start time and the times when the
compilation and execution have finished; these are displayed
by calling \p{Print\_Times}.
\item Package \p{Utilities} contains subprograms that perform string and
IO processing.
\end{itemize}

\section{Static data structures}

\subsection{Symbol table}

Package \p{Symbol\_Tables} stores the symbol table that is obtaining by
parsing data in the automata file produced by the compiler (subprogram
\p{Read}). A symbol consists of an identifier, its type and scope, the
offset in a state vector of its value, and the byte code needed to
initialize the variable.

This package exports the global variable \p{Variables} which stores the
number of variables in the \prm{} program. Within its body are tables
for numbers, strings and \p{mtype} names. The number table is used to
insert immediate operands in the \p{iconst} instructions in the byte
code (\p{Fix\_Load\_Constant}).

The channel data stores data for each channel and the global variable
\p{Channels} is exported. As in \spn{} the channels are numbered from 1,
not 0. Note the difference between a channel and a channel variable. In
the declaration:

\begin{footnotesize}
\begin{verbatim}
chan ch = [2] of { byte, byte, byte };
\end{verbatim}
\end{footnotesize}
\p{ch} is a channel variable that is just a byte holding the handle
(index) of the channel defined by the initializer. Channel variables can
be assigned to, so the association of variables with channels can change
at runtime. Buffered channels are allocated space in the state vectors,
but rendezvous channels do not take up space in the vectors.

\p{Get\_Variable\_Names} returns an array of the variable names and is
used when displaying a state. \p{Get\_Variable\_Initials} returns a vector
of the initial values after evaluating the initializing expressions.

When \p{active[N] proctype P()} is used, \p{Replicate\_Local\_Symbols} is
called to make $N-1$ copies of each local variable in this process. The
suffix \p{\_1} is added to the original symbol and \p{\_2}, \p{\_3} and
so on are added to the other copies.

Since the new copies of the local variables will be at different offsets
in the state vector, the initialization byte code must be fixed up in
the copies. This is done by creating a table \p{Byte\_Fix\_Array} whose
entries are triples: the number of the copy, the old offset and the new
offset. For example, if the first copy of byte variable \p{n} is at
offset 2 and the last (free) offset in the state vector is 7, the
next two copies will have entries: (2, 2, 7), (3, 2, 8). The procedure
\p{Fix\_Addresses} is called with the byte code for each new copy;
it scans the byte code for appropriate entries in the table and fixes
them. For example, \p{byte\_load 2 0} will become \p{byte\_load 7 0} in
copy 2 and \p{byte\_load 8 0} in copy 3.

\subsection{Automata}

Package \p{Automata} stores a data structure for the transitions, as
well as data structures for \emph{locations}, which are the possible
transitions from the set of location counters in a state.

Since the package is quite large, the subprograms for displaying
transitions and locations are placed in a child package
\p{Automata.Display}.

\subsubsection*{Declarations in the visible part}

The types declared in the package specification are:
\begin{itemize}
\item \p{Byte\_Code} is a record containing the opcode and operands.
\item \p{Byte\_Code\_Array} is an array of byte codes.
\item \p{Transitions} is a record containing the data for transitions:
source state, target state, the source code statement and its line
number, flags (atomic, end label and accept), and an array of byte codes
and the number of byte codes in the array.
Since transitions are frequently copied, for efficiency the two large
components (the statement name and the array of byte codes) are
allocated and access values used instead.
\item \p{Location\_Type} is a record with two components: a \p{Process}
index and an index of a \p{Transition} in the process.
\item \p{Location\_Record} is a record that stores sets of locations;
its components are:
\begin{itemize}
\item The locations are stored in \p{Location\_Array} of
\p{Location\_Array\_Type} whose components are of \p{Location\_Type};
\item A \p{Count} of the locations in the array;
\item \p{Never\_Index} is the index within \p{Location\_Array} of the
first transition of a never claim (counting from zero), or \p{None} is a
never claim is not used.
\end{itemize}
For example, if process \p{P} contains two transitions for its location,
process \p{Q} contains one transition, and the never claim has two
transitions. If both transitions for \p{P} and both transitions
for the never claim have the same source state, then
\p{Get\_All\_Locations} (see below) will return a \p{Location\_Record}
with:
\begin{verbatim}
Count            5
Never_Index      3
Location_Array
Process    Transition
   0           0
   0           1
   1           0
   2           0
   2           1
\end{verbatim}
\end{itemize}

There are two global variables:
\begin{itemize}
\item \p{Processes} is the number of processes in the program;
\item \p{Never} is the index of the never claim within the processes.
\end{itemize}
For example, if there are two processes and a never claim, then
\p{Processes} will be 3 and \p{Never} will be 2 (counting from zero).

The transitions are sorted with the source state as the primary key and
the target state as the secondary key, except that a transition
corresponding to \p{else} is always placed last.

For verification with fairness, a null transition is added to each
process; see Section~\ref{s.fair} for an explanation.

\subsubsection*{Declarations in the private part}

The following declarations are in the private part (so that the child
package can access them). The transitions are stored in an array
\p{Program} with one component of type \p{Process\_Type} for each
process. The components of this record type are:
\begin{itemize}
\item \p{Identifier} is the name of the process;
\item \p{Transition\_List} is an array of transitions for this process;
\item \p{Initial\_State} is the initial state of the process;
\item \p{Count} is the number of transitions in the process.
\end{itemize}

To facilitate locating accepting states (states appearing as source
states of transitions marked with the accept flag), a separate array
\p{Accept\_Table} whose components are pairs of (\p{Process}, \p{State})
is maintained. Currenly, this would not really be necessary since accept
labels are not implemented and thus accept states occur only in never
claims.

\subsubsection*{Subprograms}

The most important subprogram declared in the package is:

\begin{footnotesize}
\begin{verbatim}
  function Get_All_Locations(S: State_Vectors.State_Vector) return Location_Record;
\end{verbatim}
\end{footnotesize}

The function returns all transitions whose source state is the state
of a location counter of a process in the state vector \p{S}. For example,
in Dekker's algorithm the statement:

\begin{footnotesize}
\begin{verbatim}
    if
    :: (turn == 1)
    :: (turn == 2) -> ...
    fi
\end{verbatim}
\end{footnotesize}

gives rise to two transitions:

\begin{footnotesize}
\begin{verbatim}
number=4,source=11,target=13,atomic=0,end=0,accept=0,line=13,statement=,
  byte code=byte_load 2 0,iconst 1 0,icmpeq 0 0,
number=5,source=11,target=7,atomic=0,end=0,accept=0,line=14,statement=,
  byte code=byte_load 2 0,iconst 2 0,icmpeq 0 0,
\end{verbatim}
\end{footnotesize}
in each of the two processes. In a state where both processes are at
state 11, there will be four \emph{locations}, one for each of the two
transitions in each of the two process. Of course, only one location in each
process will be \emph{executable}. During simulation, one must be
\emph{chosen}, while during verification, both must be pushed on the
location stack so that the search can check both possibilities.

The function \p{Get\_All\_Locations} is in this package because it is
purely structural and does not depend on the runtime state; compare this
with \p{Get\_Executable\_Transitions} which does depend on the state and
thus can be found in the package \p{Execute}. Similarly,
\p{Remove\_Transition} performs the purely structural task of removing a
transition from a value of \p{Location\_Record} given an index. The
function \p{Get\_All\_Locations} can raise an exception if there are
more transitions than allowed by \p{Global.Location\_Index}.

Subprogram \p{Translate\_LTL} calls \p{LTL.LTL\_To\_Automaton} to
translate the LTL formula into a BA. The transitions of the BA are then
added as an additional process in \p{Program}.

\subsubsection*{Reading the automata file}
The ``object code'' produced by the compiler is in a form similar to
that of the output of \prg{}, except that it is not nicely ordered. The
file is parsed by procedure \p{Read}. Procedure \p{Extract\_Byte\_Code}
parses a string containing a sequence of byte codes and is exported for
parsing expressions for symbol initialization and BA transitions.
\p{Fix\_Load\_Constant} sets immediate operands in the \p{iconst}
instructions.

When \p{active[N] proctype P()} is used, the automata file contains the
number \p{N} for the process \p{P} and procedure \p{Active\_Proctypes}
is called. The transitions for this process are considered to be those
for the first copy, so $N-1$ new processes are created and the set of
transitions copied. The name of the first copy is changed to \p{P\_1}
and the copies are named \p{P\_2}, \p{P\_3} and so on.
\p{Symbol\_Tables.Fix\_Addresses} is called to fix up the addressing of
local variables, as explained above.

\section{Runtime data structures}

\subsection{State vectors}
Package \p{State\_Vectors} declares the type \p{State\_Vector} with four
components:
\begin{itemize}
\item \p{Process} is an array whose compoments are the current location
counters of the processes;
\item \p{Variable} is an array whose compoments are the current values of
the variables, as well as the contents of buffered channels;
\item \p{Inner} is a flag that is true for states in a nested search for
an acceptance cycle (this is called \p{toggle} on pp.~179--181 of
\smc{}).
\item \p{Fair} is a byte that stores the copy of the state when checking fairness.
\end{itemize}
The byte arrays of the type \p{State\_Vector} are of the constrained
subtype \p{Byte\_Array} with 256 components.

\subsection{Compressed vectors}

For storing state vectors on the stack and in the hash table,
\emph{compressed} state vectors are used, where the corresponding arrays
\p{Process} and \p{Variables} are declared with index subtypes
\p{Process\_Index} and \p{Variable\_Index} from package
\p{Config\_State}. Functions to \p{Compress} and \p{Expand} state
vectors are provided.

Type \p{Compressed\_Vector} is declared as \emph{private} in package
\p{Compress\_Vectors}. This ensures that very few packages in \prg{} are
semantically dependent on \p{Config\_State}.

\subsection{Stacks}
There are two stacks: \p{State\_Stack} for the state vectors and
\p{Location\_Stack} for the transitions to be tried in the depth-first
search. (The pseudocode in Chapter~8 of \smc{} uses
recursion instead of a location stack.) The stacks themselves are arrays
that are allocated at run-time so that their sizes can be given as
command-line arguments. Both stack packages record their maximum size
for displaying runtime statistics.

Since the state stack is declared in the package body of
\p{State\_Stack}, units that are dependent on the specification of the
package need not be recompiled if the compressed state vectors are
modified.

The function \p{On\_Stack} is used when verifying acceptance to see if a
state exists on the stack (Section~\ref{s.accept}).

The location stack stores values of type \p{Location\_Item} with
components:
\begin{itemize}
\item \p{L} of type \p{Location\_Type} (the \p{Process} and
\p{Transition} indices of the location);
\item \p{Never} is the current location within the never claim (see the discussion
of the synchronous product in Section~\ref{s.verif});
\item \p{Visited} is a flag that is set to true when a location has been
used in the depth-first search. Locations that have been visited define
the trail which is written to a file by \p{Put\_Trail} if a verification
fails;
\item The flag \p{Last} is set to true when the last location from a
state has been tried.
\end{itemize}

\subsection{Hash table}

Package \p{Hash\_Tables} uses an array of access types to linked lists
of buckets that store (compressed) state vectors. The hash function used
is FNV 1a, because it is easy to implement and efficient. See the
Wikipedia page for \emph{Fowler-Noll-Vo hash function}, or the
webpage \url{http://isthe.com/chongo/tech/comp/fnv/}.

The hash table is an access to an array whose index is a modular type,
constrained to a power-of-2 subrange of \p{2**32}. Since an access value
is used, the hash table can be allocated at run time, according to the
argument \p{-lhN}.

The procedure \p{Put\_State} returns a Boolean flag \p{Inserted} to
indicate if the state was inserted into the set or not (because it was
already there). The package maintains counts of the number of states
stored and matched, as well as a count of the numbe of collisions
encountered when hashing.

Since the package needs to access the full type declaration of
\p{Compressed\_Vector}, it is a child package of \p{Compress\_Vectors}.
An additional package, also called \p{Hash\_Tables} is declared at the
library level and simply renames the subprograms of the child package.
This is done so that the verification algorithms are not dependent on
the representation of the compressed state vectors.

\section{Execution of the model checking}

Package \p{Execute} is the parent of all the packages that perform the
simulation and verification. Its only public declaration is the
procedure \p{Run} which calls the single public declaration of either
either \p{Execute.Simulate} or \p{Execute.Verify}.
The global variables declared in the private part are:
\begin{itemize}
\item \p{Atomic} is set to the index of a process that is
executing an atomic statement. (This is the variable \p{exclusive} in
\smc{}, pp.~160--161.);
\item \p{Current} holds the current state of the simulation or
verification;
\item \p{L\_Rec} of type \p{Automata.Location\_Record} is used
to store the set of locations (transitions) from the current state;
\item \p{End\_State} of type \p{End\_Type} is used to return the end state:
valid, invalid or termination of the never claim.
\item \p{Handshake} holds the current channel index when executing a
rendezvous. (See \smc{}, pp.~555-556.)
\end{itemize}
The package body contains step counters that are incremented as each
transition is simulated or as each transition is taken during a
verification.

Procedure \p{Get\_Executable\_Transitions} returns a set of locations
that are executable in the current state. First, the set of all
locations whose source is in the current state is obtained from
\p{Get\_All\_Locations} in package \p{Automata}. Then, non-executable
transitions are removed by considering:

\begin{itemize}

\item The values of the variables in the state that might make a
transition non-executable, in which case it is removed.

\item For a buffered channel, a send statement is not executable if the
channel is full and a receive statement is not executable if the channel
is empty. For receive statements which have an argument that is not a
variable, the state vector is copied to a temporary varible and
\p{Evaluate} is called. \p{Evaluate} interprets the byte code for
loading constants and evaluating expressions (within \p{eval}), and the
receive instruction is executed, although it just modifies the temporary
state vector.

\item For a rendezvous channel, if \p{Handshake} is zero, a rendezvous
is \emph{not} in progress. If that is the case, send is executable
(provided that there is a receive statement for the same channel), and,
when it is executed, it sets \p{Handshake} to its channel index. If
\p{Handshake} is non-zero, \emph{no} statement is executable, except for
receive statements on that channel. When executed, the receive statement
then resets \p{Handshake} to zero. This method of implementing rendezvous
is described on pp.~158--160 and~555--556 of \smc{}.

\item If \p{Atomic} contains the index of a process and if there are
executable transitions for that process, all other transitions must be
removed.

\item An \p{else} transition may be taken only if no other transitions
for its process are executable; if not, the \p{else} transition is
removed.

\end{itemize}
The procedure also checks the end-state status: if there are no
executable transitions, the end state is valid if and only if the all
locations are labeled \p{end}.

Procedure \p{Execute\_At} executes a selected transition in the current
state by calling the interpreter, while procedure \p{Execute\_Never}
changes the location of the never claim.

\subsection{Interpreting statements and expressions}
Package \p{Execute.Statement} is the interpreter for the stack-based
byte code. It contains the procedure \p{Evaluate} that receives the
\p{Current} state vector and the byte code to be executed. For a
statement that stores a value, it updates the state vector. It always
returns a \p{Result} that can be used to decide if an expression is
executable.

For a rendezvous channel, when the send instruction is executed it
copies the message from the stack to an internal buffer
\p{Message\_Buffer} and sets the variable \p{Handshake} to the channel
index. The receive statement copies the message from \p{Message\_Buffer}
to the variables and resets \p{Handshake} to zero.

For a buffered channel, send pops the values of its arguments from the
stack to an array \p{Message\_Buffer}. From here it is copied to the
first position in the channel buffer in the state vector for FIFO send;
for sorted send, a search is made for the correct position to place the
message.

Receiving a value from a buffered channel is similar if the arguments
are variables. If not, the value of the argument in the stack must equal
the value of the corresponding field in a message in the channel buffer.
This is implemented my creating an array \p{Is\_Variable} of flags with
an element for each argument; it is built when the arguments of the
receive instruction are interpreted. An argument is considered to be a
variable if it is pushed onto the stack by a \p{load\_address}
instruction; other instructions that push values indicate values. When
the receive instruction is interpreted, it is now a simple matter to
check the flags in order to decide whether to copy the message field to
the variable (unless this is a poll instruction) or to check for
equality. For FIFO receive, just the first message is checked, while for
random receive, the same code is executed in a loop for all messages.
The channel buffer is ``closed up'' to remove the matched message; if a
matched message is not to be removed (\p{ch?<args>} or \p{ch?[args]}),
this step is skipped.

\subsection{Byte code instructions}
Adding a new byte code instruction requires that the following
modifications be done in the model checker (in addition to whatever
changes are done in the compiler):
\begin{itemize}
\item Write code for interpreting the instruction in an alternative of
\p{case} statement in procedure \p{Execute.Statement.Evaluate}.
\item In procedure \p{Execute.Execute\_At}, make the appropriate changes
to the \p{case} statement that decides whether the statement in which
the instruction appears needs to change the location counter in the
\p{Current} state vector. (It will not need to do so if the statement is
not executable.) The procedure assumes that this decision can
be made by examining the final instruction of a statement.
\end{itemize}
There are some other places where specific instructions are examined:
\begin{itemize}
\item In \p{Automata.}\verb+"<"+, the instruction \p{logic\_else} is
used to order the transitions so that \p{else} statements are the last
transition for any source state.
\item In \p{Automata.Fix\_Load\_Constant} and
\p{Symbol\_Tables.Fix\_Load\_Constant}, \p{iconst} instructions for 
pushing a constant on the stack are modified so that the constant value
is an immediate operand. The is done for efficiency so that the number
table need not be read for this instruction.
\item When there are multiple instantiations of a \p{proctype} that
contains local variable, \p{Symbol\_Tables.Fix\_Addresses} fixes
addresses that refer to these variables so that the correct variable for
each copy of the process is used.
\end{itemize}

\section{Simulation}
Package \p{Execute.Simulate} performs a simulation.
\p{Get\_Executable\_Transitions} from the parent package \p{Execute} is
called and one executable transition is chosen according to the options:
random, interactive or guided. Procedure \p{Execute\_At} is called to
execute the chosen transition. Exception \p{Termination\_Error} is raised
when the simulation terminates, either normally in a valid end state, or
abnormally because of errors like an invalid end state or a false
assertion. The exception is handled and the reason for the termination
displayed.

\section{Verification}\label{s.verif}
Package \p{Execute.Verify} is the parent of two child packages that
contain the code for verification in safety mode and acceptance mode
(with or without fairness).

The visible part of the package contains the single subprogram
\p{Verify}, while the private part contains subprograms that are used in
both modes of verification.

Function \p{Is\_Accept\_State} returns true if some process is in a
state that is accepting.

\p{Get\_And\_Push\_Transactions} calls \p{Get\_Executable\_Transactions}
from the parent package. If: (1) there are no executable transactions,
or (2) the only executable transactions are from the never claim, or (3)
the never claim is terminated, then the state stack is popped (after
checking the end state in safety mode). Otherwise,
\p{Push\_All\_Transitions} pushes all locations in the location record
onto the location stack. The last location (which is pushed first) has
\p{Last} set to true so that we know when to terminate the search in
this state. All locations have \p{Visited} set to false.

When there is a never claim, the \emph{synchronous product} must be
pushed: Each executable location of the program is pushed once for each
executable transition of the claim.

The private part also contains the declaration of the variable
\p{Unfolded} that is used when verifying with fairness; see
Section~\ref{s.fair} below.

\subsection{Safety}
Package \p{Execute.Verify.Safety} performs the verification of safety
properties. The main loop makes the top of the state stack the current
state and the top of the location stack the current location. If this
location has been \p{Visited}, it is popped; if it was the
\p{Last} location of this set of locations, the state stack can be
popped. Otherwise, the transition at this location is executed, as is a
transition of the never claim (if any). An attempt is made to put the
resulting state into the hash table. If successful (that is, if this is
a new state), the depth-first search is implemented by pushing the state
on the state stack and calling \p{Get\_And\_Push\_Transitions}.

\subsection{Acceptance}\label{s.accept}

Package \p{Execute.Verify.Acceptance} performs the verification of
liveness properties by looking for acceptance cycles. The overall
structure is similar to that for verification of safety, except that
when an accepting state is encountered an inner search is commenced.

Consider the the following state diagram:

\begin{center}
\unitlength=1.2pt
\begin{picture}(70,100)
%\put(0,0){\framebox(70,100){}}
\put(10,0){
\put(10,10){\circle{20}}
\put(0,0){\makebox(20,20){\textsf{s3}}}
\put(10,50){\circle{20}}
\put(0,40){\makebox(20,20){\textsf{s2}}}
\put(10,90){\circle{20}}
\put(0,80){\makebox(20,20){\textsf{s1}}}
\put(50,50){\circle{20}}
\put(50,50){\circle{18}}
\put(40,40){\makebox(20,20){\textsf{s4}}}
\put(10,80){\vector(0,-1){20}}
\put(10,40){\vector(0,-1){20}}
\put(18,83){\vector(1,-1){25}}
\put(42,43){\vector(-1,-1){25}}
\put(0,10){\line(-1,0){10}}
\put(-10,10){\line(0,1){80}}
\put(-10,90){\vector(1,0){10}}
}
\end{picture}
\end{center}

Suppose that the search begins \textsf{s1} $\rightarrow$ \textsf{s2}
$\rightarrow$ \textsf{s3}. Now \textsf{s1} has already been visited, so
the search backtracks and tries \textsf{s4}, an accepting state. The
next transition is \textsf{s4} $\rightarrow$ \textsf{s3}, but the state
\textsf{s3} has already been visited so the search terminates, although
there is an acceptance cycle \textsf{s4} $\rightarrow$ \textsf{s3}
$\rightarrow$ \textsf{s1} $\rightarrow$ \textsf{s4}.

When an accepting state like \textsf{s4} is encountered, it is pushed on
the stack with \p{Inner} set to 1 and a new search is begun. Now, the
states \textsf{s3} and \textsf{s1} \textit{with} \p{Inner} \textit{set
to 1} have \emph{not} been visited and the acceptance cycle will be
found. The accepting state from which the inner search is started is
saved in \p{Seed} and must occur again for an acceptance cycle to exist.
Alternatively, function \p{On\_Stack} is called to see if a new state
already exists on the stack; if so, a path to the seed must exist. See
pp.~179--181 of \smc{} or Section~4.4 of Baer and Katoen for more
detail.

\subsection{Fairness}\label{s.fair}
Verification with weak fairness is implemented by \emph{unfolding} the
state space as described on pp.~181--188 of SMC. Copies of states are
used to ensure that an acceptance cycle executes a transition of each
enabled process. The copies are indexed from $0$ to $k+1$, where $k$ is
the number of processes not counting the never claim; the value $k+1$ is
stored in the variable \p{Unfolded}. Each state vector has a counter
\p{Fair} (initialized to $0$) that is used to keep track of which copy
the state belongs to, and only copy $0$ has accept states.

When an \emph{accepting state} is encountered in copy $0$ of a state,
the current state becomes copy $1$ of the state; transfer from copy $i$
to copy $i+1$ occurs whenever a transition of process $i$ is executed.
The inner search is begun from the $k+1$'st copy. Since all transitions
from accepting states (in copy $0$) lead out of the copy and since the only
way of returning to copy $0$ is by traversing all the other copies, an
accepting cycle must contain transitions from all processes and thus is
fair.

When checking fairness, null transitions for blocked processes must be
added when in the unfolded states (p.~183 of \smc{}). A dummy transition
is created (by \p{Automata.Read}) at the end of the transitions for each
process (except the never claim); this transition is ignored when
executing \p{Get\_All\_Locations} in package \p{Automata}.
\p{Get\_And\_Push\_Transitions} of package \p{Execute.Verify} calls
procedure \p{Add\_Null\_Transition}, which checks if there are
executable transitions for the copy associated with the current value of
the fairness counter (not including the $0$'th and $k+1$'st copies). If
not, a null transition is constructed by calling
\p{Set\_Null\_Transition} in package \p{Automata} which sets the source
and target states to be the current state and returns the transition
index. The null transition is then added to the list of executable
transitions before the transitions of the never claim.

\section{Translation of LTL formulas to B\"{u}chi automata}\label{s.ltl}
The algorithm implemented is from:
\begin{quote}
Gerth, R., Peled, D., Vardi, M. Y., and Wolper, P. 1996. Simple
on-the-fly automatic verification of linear temporal logic. In
\emph{Proceedings of the Fifteenth IFIP WG6.1 International Symposium on
Protocol Specification, Testing and Verification XV}, 3-18.
\end{quote}
See also Section~4.4 of Kr\"{o}ger and Merz, or Section~5.2 of Baer and
Katoen.

To make the algorithms easier to follow, the unary temporal operators
\p{<>} and \p{[]} are directly implemented, rather than being translated
into \p{U} and \p{V}.

Package \p{LTL} is the parent for a set of \emph{private} packages that
implement the algorithm. Formulas and nodes are declared in the private
part of package \p{LTL}, along with sets of formulas and nodes that are
implemented using \p{Ada.Containers.Ordered\_Sets}.

A formula represented as a string is translated into a tree data
structure. For unary operators \p{!}, \p{[]} and \p{<>}, the \emph{left}
subtree is null; this is so that in both forms of future formulas
(\p{<>q} and \p{pUq}), the subformula to be fulfilled is in the right
subtree. Sets of formulas are implemented by ordered sets whose
\p{Element\_Type} is \p{Formula\_Pointer}, a pointer to the tree
representation of a formula.

The visible part of package \p{LTL} declares a \emph{function}
\p{LTL\_To\_Automaton} which takes a formula represented as a string as
a parameter and returns a set of transitions representing the BA. There
is also a \emph{procedure} of the same name that calls the function and
discards the set of transitions; it is used when the execution mode is
\p{-b}, LTL-to-BA translation only. The function first calls
\p{Push\_Negation} to push negation inwards. Function
\p{Construct\_Tableau} is called and returns the set of nodes that are
states. Finally, \p{Convert} is called to convert the states into
transitions.

Package \p{LTL.Formulas} declares subprograms for LTL
formulas. There are procedures to display formulas and sets of formulas.
\p{To\_Formula} translates a string into a formula and
\p{Get\_Literal} returns the string representation of a literal.
Function \p{Push\_Negation} returns the formula after pushing negations
inward and function \p{Contradiction} checks a set of formulas for
contradictions and returns a Boolean value. Procedure \p{Decompose}
decomposes a formula according to the rules of LTL:
one or two subformulas that must be true in the current state, and---for
the temporal operators---suformulas that must be true in the next state.

Package \p{LTL.Nodes} constructs the tableau as described in the
article. Each formula in a node is decomposed into its subformulas; when
no more decomposition is possible, either the node is closed because of
a contradiction or it is open or a new node must be created because
there are formulas that must be true in the \p{Next} state. If a new
node has the same formulas as one that already exists, it is not
created; instead, the current node is added to the list of \p{Incoming}
nodes of the existing node.

Package \p{LTL.Automaton} contains the function \p{Convert} to convert
the states into transitions. The \p{OldF} field of each state contains
the formulas that must be true in that state. The transitions are
defined as follows: for every state $s_{i}$ in the \p{Incoming} set of
state $s$, there is a transition $s_{i}\rightarrow s$. Any literal in
\p{OldF} of $s$ is a condition of the transition.

The resulting BA is optimized as follows: if there is a set of states
$s_{1}, s_{2}, \ldots$ such that they all have transitions to the same
state $t$ and labeled by the same condition, then the set can be collapsed.
This is performed in procedure \p{Optimize}, which calls
\p{Remove\_Duplicates} to collapse duplicate transitions that result
from collapsing the states.

Function \p{Get\_Future\_Formulas} returns the set of all formulas F of
the form \p{<>q} or \p{pUq}. An accepting state is one such that F is
not in the state or \p{q} is in the state.

If there is more than one future formula, the BA that results is called
a \emph{generalized B\"{u}chi} automaton with multiple sets of accepting
sets, one for each future formula. The generalized BA must be
\emph{degeneralized} to a BA with one set of accepting states. The
algorithm, described in Baer and Katoen, pp. 195--196, is implemented in
procedure \p{Degeneralize}. Copies of the states are made, one for each
future formula. The only transitions that move between copies are those
from an accepting state for the $i$'th formula; they have a target in
the next copy (modulo the number of copies). Only states in the first
copy retaining their status as accepting. It follows that if there is an
accepting cycle, it must have traversed all the copies and hence
accepting states for all future formulas.

\newpage

\section{Compiler}
\textbf{This section was written by Trishank Karthik Kuppusamy.}

This Section is an overview of the compiler. Further details can be
found in additional documents that the author will publish later.

There are four major packages in the compiler, each of which
corresponds to a phase of the compiler: the lexical analyzer
(\p{Lexer}), the recursive descent parser (\p{Parser}), the abstract
syntax tree analyzer (\p{AST}) and the code generator (\p{CodeGen}).

Each of these major packages has a corresponding child package where
their data structures are declared: \p{Types.Lexer}, \p{Types.Parser},
\p{Types.Ast}, \p{Types.Codegen}; their parent package is \p{Types}. The
\p{Types} packages make extensive use of the Ada 2005 container library:
doubly linked lists, hashed maps, vectors and hashed sets. They are used
for various book-keeping tasks such as managing the list of tokens, the
map from keywords to tokens, the string table, and the number table.

The \p{Logger} package writes messages to four output files, one for
each phase of the compiler. The logger is modeled after the standard
logger of Python with five levels of messages: debug, info, warning, error,
critical.

\subsection{Lexical analyzer}

The lexer is a 22-state deterministic finite automaton. It analyzes the
character stream and returns the appropriate token with at least the
token type. If the token is an identifier, the bounded string cursor has
a non-null value. If the token is a number, the natural value cursor has
a non-null value. The lexer uses \emph{variable buffering}: it reads a
constant number of lines into an unbounded string that has a variable
number of characters.

The lexer can also be used to compile code represented as a string. This
is used for compiling the expressions in the B\"uchi automaton as
described above in Section~\ref{s.ltl}.

\subsection{Recursive descent parser}

An LL(N) recursive descent parser is used. The \emph{First} and
\emph{Follow} sets for the left-recursion-eliminated and left-factored
Promela grammar were produced by a program previously developed by the
author. The lexer and parser can compile the entire Promela grammar,
except for embedded C code and a few other incompatibilities (such as
the use of the semicolon as both a separator \emph{and} a terminator,
and a size specifier for the \p{unsigned} data type); these are not
documented in the Promela grammar specification but were found in
\smc{}. Most of the procedures or functions in the parser package
correspond directly to a Promela construct, and most of the functions
return a node in the abstract syntax tree.

\subsection{Semantic analyzer}

The main data structures of the semantic analyzer are the nodes of the
abstract syntax tree; others include book-keeping data structures, such
as one used to distinguish between built-in data types and user-defined
types, and one for tracking \p{mtype} enumerated constants for
``constant folding.''

The semantic analyzer checker the semantic rules of the Promela
language, for example, that no \p{proctype} has more than 255 instances.
A common semantic check ensures that certain nodes are not \p{null}.
Binary expressions are not rigorously type-checked since the Promela
specification dictates that all arithmetic expressions are ultimately
computed and stored in an integer, the widest numeric type, and only
then truncated as necessary to fit the target data type. (The byte code
specification takes care of this.) Other semantic checks include
preventing assignments into a \p{mtype} constant, correcting unary
expressions, and preventing duplicate entries in the same scope.

Some tree nodes are either decorated or rewritten at compile-time. An
example of the former is that variable nodes are decorated with type
information. An example of the latter is that binary expressions are
parsed without taking operator precedence into account; instead, they are
recursively corrected for that during the semantic analysis.

\subsection{Code generator}

The generator performs two tasks: it outputs a finite state machine for
each process as represented by a set of transitions and it outputs byte
code for the Promela statement associated with each transition.

The specification of the byte code was inspired by that of the Java
Virtual Machine. The byte code instructions can be categorized as:
arithmetic, memory manipulation, input/output, bitwise/logical/boolean
operations, and control instructions. There are no explicit branching
instructions since the transitions \emph{are} the branches.

At compile-time, the set of transitions for a process may contain
\emph{noop} (no operation) transitions. This simplifies the compilation
process, because we do not have to treat certain Promela
statements---such as repeating a loop---differently while constructing
the transitions. The \emph{noop}s are eliminated after the construction
of the entire set of transitions. Atomic transitions also need
correction since \spn{} expects that the final transition in an atomic
sequence not be marked as atomic.

\subsection*{References}
Alfred V. Aho, Monica S. Lam, Ravi Sethi, \& Jeffrey D. Ullman.
\textit{Compilers: Principles, Techniques \& Tools. Second Edition}.
Pearson, 2007.

Tim Lindholm \& Frank Yellin. \textit{The Java Virtual Machine
Specification.  Second Edition.} 
\url{http://java.sun.com/docs/books/jvms/second_edition/html/VMSpecTOC.doc.html}.
\end{document}
